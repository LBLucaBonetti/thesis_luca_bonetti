\chapter{L'architettura}\label{c:architecture}

Come anticipato nel capitolo \ref{c:project}, l'architettura del sistema progettato segue il modello API RESTful.
Questa tipologia di paradigma parte dal concetto di interazione client/server e comprende una netta suddivisione tra gli aspetti legati alla logica di presentazione dei dati e quelli legati alla loro elaborazione e memorizzazione.
Nel complesso, questa separazione si concretizza tipicamente nella produzione di due progetti quasi completamente indipendenti.
L'unico nesso tra queste due parti, chiamate frontend per la presentazione e backend per elaborazione e memorizzazione, è l'interfaccia di comunicazione.
Il protocollo HTTP funge da unico collante per la trasmissione dei dati in entrambi i sensi.
\\
La struttura teorica prevede la costruzione di differenti punti di contatto, detti endpoint, che il frontend richiama quando necessita di uno o più dati.
Questi punti di contatto possono restituire informazioni, come nel caso di chiamate HTTP di tipo GET, possono memorizzarne, come nel caso di chiamate HTTP di tipo POST, possono modificare
la situazione esistente, come nel caso di chiamate HTTP di tipo PUT o PATCH e possono eliminare entità persistenti, attraverso chiamate HTTP di tipo DELETE.
Va precisato che il mapping tra funzionalità e verbi HTTP non rende di per sé una API RESTful.
\\
In un'architettura di tipo client/server, un server espone servizi verso uno o più client che lo interrogano per usufruire dei servizi esposti.
Il server è responsabile della manutenzione della sessione di ogni client che comunica con esso.
La sessione rappresenta sostanzialmente l'insieme dei dati che servono a identificare l'utente e a distinguere le richieste consentite in base a quelle precedenti.
REST, acronimo di REpresentational State Transfer, non sfugge a questa definizione ma definisce una netta differenza.
Il paradigma è stato introdotto nella tesi di dottorato di Roy Fielding \cite{fielding2000architectural}, celebre informatico statunitense contemporaneo.
Rispetto al tradizionale concetto di stato persistente mantenuto all'interno della parte server dell'applicativo, questo nuovo approccio definisce la comunicazione verso il server come stateless, ovvero senza stato.
In quest'ottica, un client che intenda usufruire di un servizio del server dovrà fornire ad esso tutti i dati necessari perché la propria richiesta sia soddisfatta.
Il server, d'altro canto, non avrà alcuna memoria rispetto alle differenti richieste giunte dal client, da cui la definizione di stateless (senza stato).
Il principale vantaggio di non avere una sessione mantenuta nel backend è un enorme alleggerimento nella gestione di client differenti e/o non omogenei.
Il ruolo del client, tuttavia, subisce un netto cambiamento: la sessione deve ora essere immagazzinata, infatti, in questa porzione dell'applicativo.
A partire dalla dissertazione con cui Fielding introdusse nel 2000 questo nuovo stile architetturale, REST è diventato uno standard de-facto per i più recenti servizi internet.
Il progressivo arricchimento delle risorse disponibili ai client ha reso la manutenzione di una sessione progressivamente meno esosa in termini di costi di elaborazione e ha sicuramente favorito la diffusione del pattern.
Un ulteriore aspetto che ha permesso a diverse aziende di adottare REST come metodologia di sviluppo standard per sistemi distribuiti e non è l'indipendenza tra frontend e backend.
Fin tanto che l'interfaccia tra le due parti non viene modificata, infatti, i due progetti possono evolvere indipendentemente l'uno rispetto all'altro e possono essere realizzati da team differenti o perfino da società diverse.
\\
