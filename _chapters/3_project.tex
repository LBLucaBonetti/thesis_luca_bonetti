\chapter{Il progetto}\label{c:project}

In un contesto aziendale come quello descritto nel capitolo \ref{c:company}, è necessaria un'organizzazione del lavoro precisa, puntuale e soprattutto condivisa da parte dei diversi team.
È infatti impensabile che differenti unità di lavoro non condividano mezzi efficaci per le attività interdipartimentali come la pianificazione e la gestione dei compiti.
Un ulteriore importante aspetto è legato alla compilazione dei rapporti d'intervento effettuati presso i clienti.
Questi costituiscono un rendiconto delle operazioni svolte su software e hardware, devono essere appositamente controfirmati da un responsabile dell'azienda presso la quale viene svolto il lavoro
e vanno consegnati all'amministrazione per l'emissione delle fatture di pagamento.
\\
INFOLOG mette a disposizione dei propri dipendenti diverse tecnologie a supporto della stesura di tali documenti:
\begin{itemize}
    \item Un software locale offline
    \item Un modulo stampabile e compilabile a mano
    \item Una parte di Unitegy che consente di esportare un file in formato PDF
\end{itemize}
Il progetto proposto si pone l'obiettivo di unificare le diverse modalità con cui vengono compilati i rapporti d'intervento, integrandosi coi sistemi aziendali esistenti.
Un'unica piattaforma che riunisca tutti i documenti emessi consentirà all'amministrazione una maggiore autonomia e faciliterà l'emissione delle fatture, consentendo una progressiva
migrazione dei dati esistenti di diversa natura, una progressiva dismissione dei vecchi servizi e fornirà altresì un singolo punto dal quale reperire le informazioni.

\section{I requisiti}

Le esigenze di modernità e di fruibilità del sistema hanno portato alla decisione di realizzare un portale web fruibile da browser su differenti dispositivi, eventualmente
convertibile in app per smartphone in un secondo momento.
Dovranno essere messe in campo diverse integrazioni per consentire a tutti gli utenti interni all'azienda di poter fruire dei servizi a loro dedicati.
A tale scopo, dovrà essere integrato il servizio di autenticazione LDAP fornito da Microsoft Active Directory, presente all'interno dei server aziendali e gestito dal reparto IT.
I programmatori del reparto Logistics, inoltre, dovranno essere in grado di utilizzare il software Atlassian Jira tramite il portale;
questo particolare programma consente una migliore gestione dello sviluppo attraverso la frammentazione delle caratteristiche da implementare in unità semplici e permette, attraverso
un vasto panorama di plug-in installabili, di gestire aspetti avanzati come le ore uomo impiegate su un singolo compito.
Gli utenti Logistics avranno la possibilità di inserire direttamente i task di Atlassian Jira nei rapporti d'intervento, annettendo automaticamente le informazioni relative alle ore uomo.
Per il resto degli utenti, sarà invece possibile l'inserimento manuale di suddetti dati.
Dovrà essere consentita la registrazione delle firme dei clienti e dei dipendenti tramite tablet o dispositivo idoneo; un utente della piattaforma potrà inserire una firma di default
da poter applicare direttamente al rapporto d'intervento senza avere la necessità di immetterla di volta in volta.
Dovrà essere possibile l'invio del rapporto compilato a un particolare indirizzo email, sia esso quello del cliente o dell'amministrazione interna o esterna, tramite un account di
Microsoft Office 365 opportunamente configurato e attraverso il protocollo SMTP.
I dati dei clienti per la compilazione dei rapporti dovranno essere letti direttamente da una particolare vista messa a disposizione dai programmatori di Unitegy; allo stesso modo, 
i rapporti compilati dovranno essere scritti e dunque inseriti direttamente in un'altra vista dello stesso database Oracle.
Infine, il progetto dovrà supportare un proprio database PostgreSQL integrato, per memorizzare i rapporti d'intervento compilati in modo centralizzato.
Questo offrirà ridondanza all'intero processo di gestione dei report, perché se uno tra i sistemi Unitegy e quello del progetto dovesse subire danneggiamenti ai dati, l'altro
offrirebbe comunque un backup allineato degli stessi.
\\
Lo stack tecnologico scelto per la produzione del software rispecchia le moderne tendenze in ambito enterprise.
Data la pregressa esperienza di INFOLOG in merito, è stata scelta la realizzazione di una API sul modello RESTful con Spring Boot per esporre i servizi ed elaborare i dati
e di un applicativo Single Page Application con Angular per la parte grafica.