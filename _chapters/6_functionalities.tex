\chapter{Le funzionalità implementate}\label{c:functionalities}

ReportManager è stato realizzato con un approccio basato su singole unità di lavoro da realizzarsi in tre mesi.
Durante il primo mese, a seguito dell'introduzione in azienda e alla raccolta delle esigenze di progetto, ci si è concentrati sulla scelta e sul successivo apprendimento delle basi
delle due tecnologie impiegate: Angular e Spring Boot.
È stato necessario un cambio di paradigma fin da subito perché durante i corsi frequentati all'università non ci si era mai soffermati sull'uso delle API RESTful per il
backend; la metodologia vista durante le lezioni era sempre stata quella che viene oggi definita come Server Side Rendering (SSR), nella quale un framework di backend si occupa
di elaborare anche la parte di presentazione grafica dei dati tramite pagine HTML che vengono poi eventualmente aggiornate dinamicamente pur restando interamente prodotte
dal server dell'applicativo che tiene traccia anche della sessione utente.
Sciolti i dubbi architetturali, il secondo mese ha visto la realizzazione di mockup iniziali che schematizzassero i flussi informativi da instaurare e della parte più consistente
del progetto.
Il terzo mese è stato dedicato infine a ultimare le caratteristiche volute e ad aggiustare dettagli di usabilità e di interazione tra frontend e backend.
Al termine del periodo, è stato eseguito un deploy locale di test su macchina virtuale per verificare l'effettivo funzionamento delle dinamiche costruite.
Il deploy è la fase di distribuzione del software, l'ultimo passo per rendere produttivo quanto realizzato.
Solitamente viene eseguito da reparti specializzati in collaborazione con gli sviluppatori ma date le finalità del lavoro in oggetto è stato scelto di simularlo.
\\
Il primo scoglio da superare è stato integrare il sistema aziendale di autenticazione dei dipendenti con la piattaforma.
In un primo momento sono stati organizzati incontri con i tecnici del settore IT che si occupano della manutenzione interna di tale infrastruttura; dalle riunioni, è
emerso che la tecnologia di riferimento utilizzata è Microsoft Active Directory.
Questo sistema utilizza il protocollo Lightweight Directory Access Protocol (LDAP) per gestire e memorizzare i dati di tutto il personale di INFOLOG.
Vengono immagazzinati in una struttura gerarchica i nomi degli utenti, le password di accesso e i permessi di lettura e scrittura sulle cartelle della rete locale condivisa internamente.
Si tratta dunque di uno standard per gestire completamente e in modo centralizzato quali operazioni sono consentite alle diverse figure e ai diversi ruoli tra gli impiegati.
Per certi versi, un sistema di naming (così viene definito) come LDAP condivide caratteristiche col sistema di risoluzione dei nomi DNS che si utilizza quotidianamente per associare
un indirizzo IP a una stringa URL di una risorsa sul web durante la navigazione.
Nel caso specifico di INFOLOG, le risorse di interesse sono gli utenti che faranno parte di ReportManager e l'identificazione degli stessi si ottiene dal nodo della gerarchia
identificato dai parametri standardizzati dalla specifica denominata X.500; in particolare:
\begin{itemize}
    \item OU: unità organizzativa
    \item DC: componente di dominio
    \item DC: sottocomponente di dominio
\end{itemize}
Messi a fuoco questi tre punti, è stato ricercato un componente di Spring Boot che fornisse una API chiara e semplice per integrare questo tipo di autenticazione.
La scelta è ricaduta sul componente ActiveDirectoryLdapAuthenticationProvider di Spring Security, già incluso in fase di inizializzazione del progetto backend.
Attraverso il costruttore di questa classe, sono stati specificati i parametri di cui sopra come root degli utenti e sono state incluse le informazioni
relative all'indirizzo pubblico del sistema Active Directory della rete da contattare per verificare le credenziali e la presenza degli utenti, oltre al nome di dominio.
Terminata la configurazione di questo componente, diventato il cosiddetto AuthenticationProvider del progetto e cioè quel componente che gestisce in automatico
l'autenticazione di un utente al momento della richiesta verso il backend, sono stati necessari ulteriori passi per risolvere alcune problematiche di diversa natura.
Anzitutto è stata abilitata la gestione del CORS, acronimo di Cross-Origin Resource Sharing.
Il CORS è un meccanismo di difesa realizzato per impedire a qualunque origine non conosciuta di utilizzare le funzionalità esposte dal backend.
Un'origine, nel contesto, è una qualsiasi combinazione di tre parametri dello stack TCP/IP standard: dominio, schema e porta.
Nel caso di ReportManager, l'unica applicazione in grado di comunicare e condividere risorse con la parte server deve essere l'applicazione Angular del frontend.
Ci si è poi posti il problema di come mantenere lo stato autenticato per un utente nel frontend.
Nel backend questa caratteristica è delegata a LDAP; questa parte dell'applicazione concettualmente lavora per singole richieste e in quanto RESTful non ha il concetto di stato.
Lo stato, infatti, è mantenuto nel frontend e dunque si rende necessario un meccanismo efficace di gestione di login e logout.
Come da best practice diffusasi con l'avvento di REST stesso, la scelta in questo caso è ricaduta sui JSON Web Tokens (JWT).
Un JSON Web Token è un insieme di dati sotto forma di oggetto JavaScript che il frontend custodisce e invia in un particolare header HTTP con ogni chiamata al backend.
Se l'utente è riconosciuto e ha il permesso di eseguire operazioni è perché nella chiamata ha annesso questo token.
Al suo interno sono infatti presenti le informazioni specifiche su ciò che può o non può fare un determinato soggetto: le claim.
Oltre a queste, è specificata una data di scadenza che serve come ulteriore misura di sicurezza per il caso in cui un utente malintenzionato riuscisse a impossessarsi del token.
Perché il meccanismo stia in piedi e non ci siano forzature o compromissioni, è necessario che ogni JWT sia correttamente verificato e approvato dal backend.
Nel caso del progetto ReportManager, la sicurezza è garantita dalla firma digitale apposta sul JWT stesso; questa firma è ottenuta e verificabile tramite una coppia di chiavi
(pubblica e privata) che costituiscono un'implementazione del concetto di cifratura asimmetrica.
Riassumendo, un dipendente che voglia effettuare il login col proprio account Active Directory (LDAP) aziendale compirà implicitamente i seguenti step:
\begin{itemize}
    \item Inserirà le proprie credenziali e sottometterà le stesse tramite form presentata dal progetto Angular di frontend all'avvio
    \item Il backend le riceverà a un particolare endpoint che verificherà in primis la correttezza della sorgente CORS
    \item Il backend verificherà poi la presenza del JWT e la sua validità sulla base delle chiavi definite
    \item Le informazioni saranno inviate a LDAP per l'effettiva validazione dell'utente tramite i parametri configurati di cui sopra
    \item Se non presente nel database locale di ReportManager, l'utente sarà aggiunto a esso per non richiedere i dettagli a LDAP a ogni richiesta successiva
    \item Verrà prodotto un nuovo JWT per il richiedente e sarà restituito al frontend che lo salverà localmente per rendere finalmente operativo il soggetto
\end{itemize}
Nel capitolo \ref{c:project} è stato anticipato che i dipendenti di INFOLOG che utilizzeranno ReportManager si dividono in due categorie: quelli che fanno parte della piattaforma
Atlassian Jira e quelli che non ne fanno parte.
A seguito di un'attenta analisi del flusso descritto poc'anzi, può sembrare che manchi un dettaglio importante: come distinguere un utente Jira da uno Unitegy "semplice"?
È infatti vero che un qualunque impiegato di INFOLOG fa parte di Unitegy mentre solo la porzione di questi che lavora in Logistics utilizza il software di Atlassian.
In realtà il token JWT fornirà al frontend anche questa informazione; dopo l'interrogazione a LDAP per sapere se presente davvero in INFOLOG, è stata implementata infatti
una chiamata a un'API REST esposta da Jira stesso che consente di verificare proprio l'esistenza di un certo utente sulla base dell'indirizzo email.
È stato quindi verificato con il reparto IT che come regola ogni utente Jira di INFOLOG accedesse effettivamente con la stessa email aziendale presente su Unitegy.
Esclusa la possibilità che i due indirizzi non coincidessero, l'integrazione è stata dunque completata.