\chapter{Le tecnologie}\label{c:technologies}

Il frontend del software è stato realizzato con Angular.
Angular è un framework di sviluppo per interfacce e applicazioni web sviluppato e mantenuto da Google.
Viene distribuito tramite licenza in stile MIT ed è dunque possibile visionare, modificare e redistribuire il codice o parte di esso senza alcun stringente limite.
La prima versione della piattaforma risale al 2010 e prende il nome di AngularJS; l'avvento di questo strumento ha consentito a moltissimi programmatori di avvicinarsi
allo sviluppo di interfacce grafiche e pagine web sempre più dinamiche grazie all'uso massivo di JavaScript per l'interattività.
Nel 2016, il software viene completamente riscritto per adattarsi alle esigenze di modernità del panorama di sviluppo web emergente e più recente.
Il nome viene cambiato in Angular e nasce il progetto di cui ReportManager fa uso.
Il linguaggio di programmazione attualmente impiegato all'interno di questo strumento è TypeScript, un superset di JavaScript.
Un superset, nel gergo informatico, indica una tecnologia che ne estende una già esistente, arricchendone le caratteristiche e migliorandola.
Nel caso di TypeScript, la differenza sostanziale rispetto a JavaScript è legata alla tipizzazione.
Se uno dei maggiori punti a sfavore di linguaggi di natura dinamica come JavaScript e Python è l'uso di meccanismi complessi per la comprensione dei tipi delle 
variabili, soprattutto per progetti di medie e grandi dimensioni, TypeScript offre una soluzione a questo (presunto) difetto fornendo agli sviluppatori la sintassi
per introdurre un tipo di tipizzazione forte.
In realtà i vantaggi sono visibili tipicamente durante la fase di sviluppo, perché TypeScript viene poi tradotto in JavaScript nella fase di interpretazione del codice.
L'attività dello sviluppatore risulta migliore soprattutto grazie alla possibilità di definire per ogni oggetto di tipo JSON la natura di ogni parametro espresso; questo 
dà l'enorme vantaggio di non doversi affidare unicamente al nome di una variabile o all'uso nello scorrimento delle operazioni del software per comprenderne l'effettivo impiego.
\\
Angular offre numerose altre caratteristiche, oltre al TypeScript.
Nasce come multipiattaforma, rendendo possibile lo sviluppo di applicativi che possano correttamente funzionare su browser diversi come da sua natura, su sistemi desktop e su 
smartphone dotati di sistemi operativi Android o iOS grazie a interfacce come Apache Cordova.
Il sistema è stato riprogettato con velocità e performance come punti cardine, rendendo qualunque applicazione realizzata con esso estremamente rapida e professionale con 
le semplici impostazioni di default.
Moltissimi ambienti di sviluppo integrati (o IDE) offrono un'ottima compatibilità con esso e con i tool specifici che aiutano nel programmare e sfruttare al meglio le sue 
caratteristiche.
Trattandosi di un framework sviluppato fin da 2016, infine, Angular è estremamente diffuso letteralmente in tutto il mondo e ha alle spalle numerosi team che lo supportano,
che ne segnalano problematiche nella repository pubblica comune e che lo utilizzano per progetti in ambiti che spaziano dal privato all'enterprise.
La community è una delle più attive tra i numerosi progetti dello stesso tipo presenti al giorno d'oggi e si contende con React e Vue lo scettro di miglior ambiente per lo 
sviluppo di interfacce web.
\\
All'interno di ReportManager, Angular è fiancheggiato da altre librerie che meritano menzione.
Una di queste è sicuramente RxJS. Essa viene impiegata per la gestione delle chiamate asincrone e dei flussi di dati basati su eventi.
In estrema sintesi, RxJS implementa il design pattern observer nel quale un oggetto osservato funge da fulcro e tutti gli ascoltatori interessati a tale oggetto e ai dati ad esso
correlati scelgono di abbonarsi per ricevere aggiornamenti puntuali rispetto a qualsivoglia modifica.
Un'altra libreria è Angular Material, sfruttata per il duplice obiettivo di evitare la creazione di ogni singolo componente da zero e di dare un'idea di coesione stilistica
all'intera interfaccia utente pur non avendo alcuna competenza grafica pregressa.
Per quanto riguarda la gestione ottimizzata delle date e degli orari tra le viste applicative e il backend, è stata impiegata la libreria Moment.js.
Signature Pad è stata usata per gestire la firma di un dipendente o di un cliente in modo semplice e chiaro.
Infine, il package uuid è stato usato per generare id univoci ove necessario.