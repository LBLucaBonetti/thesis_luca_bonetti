\chapter{Il debito tecnico e gli sviluppi futuri}\label{c:technical_debt}

Il debito tecnico\footnote{https://martinfowler.com/bliki/TechnicalDebt.html} è un termine metaforico coniato
da Ward Cunningham\footnote{https://it.wikipedia.org/wiki/Ward\_Cunningham}, programmatore statunitense di fama
internazionale, per descrivere come l'attività di sviluppo di un software finalizzata unicamente al rilascio
porti inevitabilmente con sé ripercussioni nelle fasi successive.
\\
Prendendo in prestito un esempio\footnote{https://martinfowler.com/bliki/TechnicalDebt.html} da Martin Fowler\footnote{https://it.wikipedia.org/wiki/Martin\_Fowler}, altro famoso ingegnere informatico, si può paragonare
tale concetto a un debito finanziario; il tempo aggiuntivo che si impiegherà per aggiungere una nuova
caratteristica al progetto è da vedersi come l'interesse sul debito e idealmente bisogna cercare di farlo tendere
a zero.
\\
Sono stati condotti numerosi studi sulla questione, alcuni dei quali\cite{tom2013exploration}, propongono metodologie
di carattere volutamente generale per comprendere il fenomeno e analizzarlo.
\\
Nonostante ciò, numerosissime aziende faticano a comprendere il concetto, a riconoscerlo e di conseguenza a
fronteggiarlo.
\\
ReportManager è stato realizzato tenendo ben presente questa possibile problematica.
Ogni scelta è stata condivisa con il management, ogni dipendenza annessa è stata illustrata ai membri del team
Logistics e si è cercato di documentare il più possibile le funzionalità principali per rendere la manutenzione
agevole a chiunque dovrà approcciarvisi.
\\
Nonostante l'impegno e la buona volontà, l'applicazione non è completa in ogni sua parte.
Diverse sono le migliorie future che potranno essere programmate e messe in produzione e il seguente elenco,
sebbene non esaustivo, riassume i punti individuati:
\begin{itemize}
    \item Utilizzo di un controllo unico per verificare il login utente lato frontend anziché uno replicato su ogni pagina
    \item Miglioramento della descrizione degli errori sia nel logging del backend che nel frontend
    \item Miglioramento della posizione dei vari popup mostrati nelle differenti pagine, con codice colori rosso, giallo e verde per distinguere la gravità del messaggio
    \item Uso dello spinner di caricamento in conseguenza di ogni operazione potenzialmente lenta
    \item Eliminazione del logging di debug lato frontend
    \item Sostituzione dei parametri fissati nelle interrogazioni delle API di Atlassian Jira con i corrispondenti dinamici
    \item Miglioramento della gestione delle firme e passaggio da stringa a immagine reale
    \item Inserimento di un bottone per creare un nuovo report se attualmente non ne sono stati creati per quell'utente
\end{itemize}
Un ultimo aspetto che vale la pena di affrontare è quello relativo alle \emph{best practices}, in italiano \emph{buone pratiche}.
Le best practices sono pattern di buona programmazione o comunque di organizzazione del software che vengono
consigliate spesso dagli sviluppatori dei framework che si usano o dagli utenti più esperti.
Chiaramente si tratta talvolta di finezze e altre volte invece di concetti basilari per un occhio esperto.
Il neofita, tuttavia, trae beneficio da questi e impara la filosofia con cui i framework funzionano e come sono
stati pensati, assumendo maggiore consapevolezza del lavoro che dovrà svolgere.
ReportManager non fa uso esplicito di best practices e in una fase futura di refactoring dovrà assolutamente
prevederne l'impiego.
\\
Il mondo Angular prevede, ad esempio, di utilizzare i service per tutte le funzionalità di business logic del
software; questo aderisce perfettamente al concetto di Model-View-Controller me nel software in oggetto se ne fa
un uso solo parziale.
\\
Nel mondo Spring Boot, invece, buone pratiche di esempio possono essere il principio di singola responsabilità e
l'uso di Data Transfer Object (o DTO).
Il primo indica l'idea secondo cui ogni metodo e più in generale ogni classe sia deputata a un solo compito;
il secondo forza invece una netta separazione tra il backend e il frontend, affermando che non tutti i dettagli
mappati su database debbano essere riportati all'interfaccia e che sia sufficiente un sottoinsieme di questi, o
talvolta un insieme semplicemente differente di informazioni.
\\
Concludendo, un'applicazione moderna non può ritenersi tale senza test automatizzati. Esistono diversi strumenti
coi quali è possibile attuare tecniche di programmazione efficaci come il Test-Driven Development (TDD)
e la scrittura di codice avente per scopo il test di porzioni di programma.
Sia Angular che Spring Boot offrono un supporto estremamente ampio a questa tematica anche se in una fase iniziale
di progetto si tende spesso a tralasciare questo aspetto.