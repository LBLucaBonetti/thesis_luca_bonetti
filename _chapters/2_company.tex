\chapter{L'azienda}\label{c:company}

INFOLOG SpA è una grande realtà aziendale del territorio modenese che si occupa da almeno 25 anni di progettazione e sviluppo di soluzioni software.
L'attività nasce negli anni '80 e si incentra fin da subito sul settore gestionale e su quello logistico.
Con l'avvento del nuovo millennio, nei laboratori viene adottato il linguaggio di programmazione Java come standard e inizia una progressiva e costante crescita
che porta la ditta alla trasformazione in S.P.A., avvenuta nel 2005.
Questa espansione continua nel 2012 con l'acquisizione di Netfabrica srl, attiva nei settori di assistenza e consulenza informatica.
Nel 2020, infine, la presenza sul mercato dell'azienda si rafforza ulteriormente grazie all'ingresso in Var Group del Gruppo Sesa.
\\
Internamente, INFOLOG (da questo punto in poi sarà omessa la desinenza SpA per brevità) è strutturata in 5 reparti operativi:
\begin{itemize}
    \item L'amministrazione si occupa della gestione del personale e della sovrintendenza
    \item Il reparto tecnico comprende operatori specializzati per la gestione della strumentazione e delle reti interne e per l'installazione dei sistemi presso i clienti
    \item AS400 è la divisione che si occupa dello sviluppo e della manutenzione degli omonimi sistemi legacy
    \item Unitegy è la branca che si occupa della soluzione di gestione delle attività commerciali, basata sul software open source Compiere
    \item Logistics è il reparto più numeroso e attivo negli ambiti di ricerca e sviluppo ed è responsabile della produzione, dell'integrazione e della customizzazione di INTELLIMAG, il gestionale di magazzino proposto in ambito nazionale e internazionale
\end{itemize}
La realizzazione del progetto oggetto di questa tesi è stata svolta presso INFOLOG con l'aiuto del team Logistics.