% SOMMARIO E LINEE GUIDA

% 1) Introduzione (finalità del tirocinio a grandi linee)
% 2) L'azienda (settore di competenza, prodotti principali)
% 3) Il progetto (da dove nasce l'esigenza, quali sono gli scopi)
% 4) L'architettura (REST)
% 5) Le tecnologie (Spring Boot, Angular, ...)
% 6) Le funzionalità implementate (inserimento delle ore tramite form, inserimento di un'immagine per la firma del rapporto di lavoro, compilazione in PDF di un rapporto, ...)
% 7) Le integrazioni (il software comunica direttamente con un'API di una estensione di Atlassian Jira, con un database Oracle di uno dei prodotti interni aziendali, integra un modulo di invio email, in fase di autenticazione/autorizzazione si sfrutta LDAP della intranet aziendale, ...)
% 8) Il debito tecnico (mia mancanza di conoscenza generale, best practices, ...)
% 9) Conclusioni

\documentclass[12pt,a4paper,oneside]{book}
\usepackage[italian]{babel}
\renewcommand{\baselinestretch}{1.5} 
\usepackage[]{frontespizio}

\begin{document}

\begin{frontespizio}
\Rientro {1.5cm}
\Margini {1.5cm}{1.5cm}{1.5cm}{1cm}
\Universita {Modena e Reggio Emilia}
\Dipartimento {Scienze Fisiche, Informatiche e Matematiche}
\Corso [Laurea]{Informatica}
\Annoaccademico {2020-2021}
\Titoletto {Tesi di Laurea}
\Titolo {Una web application per la gestione dei rapporti d'intervento}
\Candidato []{Luca Bonetti}
\Preambolo{\renewcommand{\frontsmallfont}[1]{\small}}
\Relatore {Chiar.mo Prof. Giacomo Cabri}
\end{frontespizio}

\newpage
\begin{flushright}
\null\vspace{\stretch{1}}
\textit{Tesi.}
\vspace{\stretch{1}}\null
\end{flushright}

\tableofcontents

\chapter{Introduzione}\label{c:introduction}

Nel lavoro che si descrive...

Il resto del documento è organizzato come segue.
Infine, il capitolo \ref{c:conclusions} conclude il documento.

\chapter{L'azienda}\label{c:company}

INFOLOG SpA\footnote{https://www.infolog.it/} è una grande realtà aziendale del territorio modenese che si occupa da almeno 25 anni di progettazione e sviluppo di soluzioni software.
L'attività nasce negli anni '80 e si incentra fin da subito sul settore gestionale e su quello logistico.
Con l'avvento del nuovo millennio, nei laboratori viene adottato il linguaggio di programmazione Java\footnote{https://www.java.com/it/} come standard e inizia una progressiva e costante crescita
che porta la ditta alla trasformazione in S.P.A., avvenuta nel 2005.
Questa espansione continua nel 2012 con l'acquisizione di Netfabrica srl, attiva nei settori di assistenza e consulenza informatica.
Nel 2020, infine, la presenza sul mercato dell'azienda si rafforza ulteriormente grazie all'ingresso in Var Group\footnote{https://www.vargroup.it/} del Gruppo Sesa\footnote{https://www.sesa.it/}.
\\
Internamente, INFOLOG SpA (da questo punto in poi chiamata semplicemente INFOLOG per brevità) è strutturata in 5 reparti operativi:
\begin{itemize}
    \item L'amministrazione si occupa della gestione del personale, della sovrintendenza e della fatturazione
    \item Il reparto tecnico comprende operatori specializzati per la gestione della strumentazione e delle reti interne e per l'installazione dei sistemi presso i clienti
    \item AS400 è la divisione che si occupa dello sviluppo e della manutenzione degli omonimi sistemi legacy (oggi IBM i\footnote{https://www.ibm.com/it-it/it-infrastructure/power/os/ibm-i})
    \item Unitegy è la branca che si occupa della soluzione di gestione delle attività commerciali, basata sul software open source Compiere\footnote{http://www.compiere.com/}
    \item Logistics è il reparto più numeroso e attivo negli ambiti di ricerca e sviluppo ed è responsabile della produzione, dell'integrazione e della customizzazione di INTELLIMAG, il gestionale di magazzino proposto in ambito nazionale e internazionale
\end{itemize}
La realizzazione del progetto oggetto di questa tesi è stata svolta presso INFOLOG con l'aiuto del team Logistics.
\chapter{Il progetto}\label{c:project}
\chapter{L'architettura}\label{c:architecture}
\chapter{Le tecnologie}\label{c:technologies}

Il frontend del software è stato realizzato con Angular.
Angular è un framework di sviluppo per interfacce e applicazioni web sviluppato e mantenuto da Google.
Viene distribuito tramite licenza in stile MIT ed è dunque possibile visionare, modificare e redistribuire il codice o parte di esso senza alcun stringente limite.
La prima versione della piattaforma risale al 2010 e prende il nome di AngularJS; l'avvento di questo strumento ha consentito a moltissimi programmatori di avvicinarsi
allo sviluppo di interfacce grafiche e pagine web sempre più dinamiche grazie all'uso massivo di JavaScript per l'interattività.
Nel 2016, il software viene completamente riscritto per adattarsi alle esigenze di modernità del panorama di sviluppo web emergente e più recente.
Il nome viene cambiato in Angular e nasce il progetto di cui ReportManager fa uso.
Il linguaggio di programmazione attualmente impiegato all'interno di questo strumento è TypeScript, un superset di JavaScript.
Un superset, nel gergo informatico, indica una tecnologia che ne estende una già esistente, arricchendone le caratteristiche e migliorandola.
Nel caso di TypeScript, la differenza sostanziale rispetto a JavaScript è legata alla tipizzazione.
Se uno dei maggiori punti a sfavore di linguaggi di natura dinamica come JavaScript e Python è l'uso di meccanismi complessi per la comprensione dei tipi delle 
variabili, soprattutto per progetti di medie e grandi dimensioni, TypeScript offre una soluzione a questo (presunto) difetto fornendo agli sviluppatori la sintassi
per introdurre un tipo di tipizzazione forte.
In realtà i vantaggi sono visibili tipicamente durante la fase di sviluppo, perché TypeScript viene poi tradotto in JavaScript nella fase di interpretazione del codice.
L'attività dello sviluppatore risulta migliore soprattutto grazie alla possibilità di definire per ogni oggetto di tipo JSON la natura di ogni parametro espresso; questo 
dà l'enorme vantaggio di non doversi affidare unicamente al nome di una variabile o all'uso nello scorrimento delle operazioni del software per comprenderne l'effettivo impiego.
\\
Angular offre numerose altre caratteristiche, oltre al TypeScript.
Nasce come multipiattaforma, rendendo possibile lo sviluppo di applicativi che possano correttamente funzionare su browser diversi come da sua natura, su sistemi desktop e su 
smartphone dotati di sistemi operativi Android o iOS grazie a interfacce come Apache Cordova.
Il sistema è stato riprogettato con velocità e performance come punti cardine, rendendo qualunque applicazione realizzata con esso estremamente rapida e professionale con 
le semplici impostazioni di default.
Moltissimi ambienti di sviluppo integrati (o IDE) offrono un'ottima compatibilità con esso e con i tool specifici che aiutano nel programmare e sfruttare al meglio le sue 
caratteristiche.
Trattandosi di un framework sviluppato fin da 2016, infine, Angular è estremamente diffuso letteralmente in tutto il mondo e ha alle spalle numerosi team che lo supportano,
che ne segnalano problematiche nella repository pubblica comune e che lo utilizzano per progetti in ambiti che spaziano dal privato all'enterprise.
La community è una delle più attive tra i numerosi progetti dello stesso tipo presenti al giorno d'oggi e si contende con React e Vue lo scettro di miglior ambiente per lo 
sviluppo di interfacce web.
\\
All'interno di ReportManager, Angular è fiancheggiato da altre librerie che meritano menzione.
Una di queste è sicuramente RxJS. Essa viene impiegata per la gestione delle chiamate asincrone e dei flussi di dati basati su eventi.
In estrema sintesi, RxJS implementa il design pattern observer nel quale un oggetto osservato funge da fulcro e tutti gli ascoltatori interessati a tale oggetto e ai dati ad esso
correlati scelgono di abbonarsi per ricevere aggiornamenti puntuali rispetto a qualsivoglia modifica.
Un'altra libreria è Angular Material, sfruttata per il duplice obiettivo di evitare la creazione di ogni singolo componente da zero e di dare un'idea di coesione stilistica
all'intera interfaccia utente pur non avendo alcuna competenza grafica pregressa.
Per quanto riguarda la gestione ottimizzata delle date e degli orari tra le viste applicative e il backend, è stata impiegata la libreria Moment.js.
Signature Pad è stata usata per gestire la firma di un dipendente o di un cliente in modo semplice e chiaro.
Infine, il package uuid è stato usato per generare id univoci ove necessario.
\chapter{Le funzionalità implementate}\label{c:functionalities}

ReportManager è stato realizzato con un approccio basato su singole unità di lavoro da realizzarsi in tre mesi.
Durante il primo mese, a seguito dell'introduzione in azienda e alla raccolta delle esigenze di progetto, ci si è concentrati sulla scelta e sul successivo apprendimento delle basi
delle due tecnologie impiegate: Angular e Spring Boot.
È stato necessario un cambio di paradigma fin da subito perché durante i corsi frequentati all'università non ci si era mai soffermati sull'uso delle API RESTful per il
backend; la metodologia vista durante le lezioni era sempre stata quella che viene oggi definita come Server Side Rendering (SSR), nella quale un framework di backend si occupa
di elaborare anche la parte di presentazione grafica dei dati tramite pagine HTML che vengono poi eventualmente aggiornate dinamicamente pur restando interamente prodotte
dal server dell'applicativo che tiene traccia anche della sessione utente.
Sciolti i dubbi architetturali, il secondo mese ha visto la realizzazione di mockup iniziali che schematizzassero i flussi informativi da instaurare e della parte più consistente
del progetto.
Il terzo mese è stato dedicato infine a ultimare le caratteristiche volute e ad aggiustare dettagli di usabilità e di interazione tra frontend e backend.
Al termine del periodo, è stato eseguito un deploy locale di test su macchina virtuale per verificare l'effettivo funzionamento delle dinamiche costruite.
Il deploy è la fase di distribuzione del software, l'ultimo passo per rendere produttivo quanto realizzato.
Solitamente viene eseguito da reparti specializzati in collaborazione con gli sviluppatori ma date le finalità del lavoro in oggetto è stato scelto di simularlo.
\\
Il primo scoglio da superare è stato integrare il sistema aziendale di autenticazione dei dipendenti con la piattaforma.
In un primo momento sono stati organizzati incontri con i tecnici del settore IT che si occupano della manutenzione interna di tale infrastruttura; dalle riunioni, è
emerso che la tecnologia di riferimento utilizzata è Microsoft Active Directory.
Questo sistema utilizza il protocollo Lightweight Directory Access Protocol (LDAP) per gestire e memorizzare i dati di tutto il personale di INFOLOG.
Vengono immagazzinati in una struttura gerarchica i nomi degli utenti, le password di accesso e i permessi di lettura e scrittura sulle cartelle della rete locale condivisa 
internamente.
Si tratta dunque di uno standard per gestire completamente e in modo centralizzato quali operazioni sono consentite alle diverse figure e ai diversi ruoli tra gli impiegati.
Per certi versi, un sistema di naming (così viene definito) come LDAP condivide caratteristiche col sistema di risoluzione dei nomi DNS che si utilizza quotidianamente per associare
un indirizzo IP a una stringa URL di una risorsa sul web durante la navigazione.
Nel caso specifico di INFOLOG, le risorse di interesse sono gli utenti che faranno parte di ReportManager e l'identificazione degli stessi si ottiene dal nodo della gerarchia
identificato dai parametri standardizzati dalla specifica denominata X.500; in particolare:
\begin{itemize}
    \item OU: unità organizzativa
    \item DC: componente di dominio
    \item DC: sottocomponente di dominio
\end{itemize}
Messi a fuoco questi tre punti, è stato ricercato un componente di Spring Boot che fornisse una API chiara e semplice per integrare questo tipo di autenticazione.
La scelta è ricaduta sul componente ActiveDirectoryLdapAuthenticationProvider di Spring Security, già incluso in fase di inizializzazione del progetto backend.
Attraverso il costruttore di questa classe, sono stati specificati i parametri di cui sopra come root degli utenti e sono state incluse le informazioni
relative all'indirizzo pubblico del sistema Active Directory della rete da contattare per verificare le credenziali e la presenza degli utenti, oltre al nome di dominio.
Terminata la configurazione di questo componente, diventato il cosiddetto AuthenticationProvider del progetto e cioè quel componente che gestisce in automatico
l'autenticazione di un utente al momento della richiesta verso il backend, sono stati necessari ulteriori passi per risolvere alcune problematiche di diversa natura.
Anzitutto è stata abilitata la gestione del CORS, acronimo di Cross-Origin Resource Sharing.
Il CORS è un meccanismo di difesa realizzato per impedire a qualunque origine non conosciuta di utilizzare le funzionalità esposte dal backend.
Un'origine, nel contesto, è una qualsiasi combinazione di tre parametri dello stack TCP/IP standard: dominio, schema e porta.
Nel caso di ReportManager, l'unica applicazione in grado di comunicare e condividere risorse con la parte server deve essere l'applicazione Angular del frontend.
Ci si è poi posti il problema di come mantenere lo stato autenticato per un utente nel frontend.
Nel backend questa caratteristica è delegata a LDAP; questa parte dell'applicazione concettualmente lavora per singole richieste e in quanto RESTful non ha il concetto di stato.
Lo stato, infatti, è mantenuto nel frontend e dunque si rende necessario un meccanismo efficace di gestione di login e logout.
Come da best practice diffusasi con l'avvento di REST stesso, la scelta in questo caso è ricaduta sui JSON Web Tokens (JWT).
Un JSON Web Token è un insieme di dati sotto forma di oggetto JavaScript che il frontend custodisce e invia in un particolare header HTTP con ogni chiamata al backend.
Se l'utente è riconosciuto e ha il permesso di eseguire operazioni è perché nella chiamata ha annesso questo token.
Al suo interno sono infatti presenti le informazioni specifiche su ciò che può o non può fare un determinato soggetto: le claim.
Oltre a queste, è specificata una data di scadenza che serve come ulteriore misura di sicurezza per il caso in cui un utente malintenzionato riuscisse a impossessarsi del token.
Perché il meccanismo stia in piedi e non ci siano forzature o compromissioni, è necessario che ogni JWT sia correttamente verificato e approvato dal backend.
Nel caso del progetto ReportManager, la sicurezza è garantita dalla firma digitale apposta sul JWT stesso; questa firma è ottenuta e verificabile tramite una coppia di chiavi
(pubblica e privata) che costituiscono un'implementazione del concetto di cifratura asimmetrica.
Riassumendo, un dipendente che voglia effettuare il login col proprio account Active Directory (LDAP) aziendale compirà implicitamente i seguenti step:
\begin{itemize}
    \item Inserirà le proprie credenziali e sottometterà le stesse tramite form presentata dal progetto Angular di frontend all'avvio
    \item Il backend le riceverà a un particolare endpoint che verificherà in primis la correttezza della sorgente CORS
    \item Il backend verificherà poi la presenza del JWT e la sua validità sulla base delle chiavi definite
    \item Le informazioni saranno inviate a LDAP per l'effettiva validazione dell'utente tramite i parametri configurati di cui sopra
    \item Se non presente nel database locale di ReportManager, l'utente sarà aggiunto a esso per non richiedere i dettagli a LDAP a ogni richiesta successiva
    \item Verrà prodotto un nuovo JWT per il richiedente e sarà restituito al frontend che lo salverà localmente per rendere finalmente operativo il soggetto
\end{itemize}
Nel capitolo \ref{c:project} è stato anticipato che i dipendenti di INFOLOG che utilizzeranno ReportManager si dividono in due categorie: quelli che fanno parte della piattaforma
Atlassian Jira e quelli che non ne fanno parte.
A seguito di un'attenta analisi del flusso descritto poc'anzi, può sembrare che manchi un dettaglio importante: come distinguere un utente Jira da uno Unitegy "semplice"?
È infatti vero che un qualunque impiegato di INFOLOG fa parte di Unitegy mentre solo la porzione di questi che lavora in Logistics utilizza il software di Atlassian.
In realtà il token JWT fornirà al frontend anche questa informazione; dopo l'interrogazione a LDAP per sapere se presente davvero in INFOLOG, è stata implementata infatti
una chiamata a un'API REST esposta da Jira stesso che consente di verificare proprio l'esistenza di un certo utente sulla base dell'indirizzo email.
È stato quindi verificato con il reparto IT che come regola ogni utente Jira di INFOLOG accedesse effettivamente con la stessa email aziendale presente su Unitegy.
Esclusa la possibilità che i due indirizzi non coincidessero, l'integrazione è stata dunque completata.
\chapter{Le integrazioni}\label{c:integrations}
\chapter{Il debito tecnico}\label{c:technical_debt}
\input{_chapters/9_conclusions.tex}

\appendix

\bibliographystyle{plain}
\bibliography{thesis}

\end{document}
